\section{Introduction}
A desire for robotic solutions, particularly in the small and medium scale enterprises (SMEs) is becoming increasingly prominent. Automation and robotics promise to deliver reduction on production costs and increase in productivity. However, traditional automation implies an investment prohibitive for SMEs, whose activities mainly involve small batches of production and high variety of products, for example, due to a seasonal nature of their operations. Concretely, tasks such as assembly, machine filling or packaging, can be automated with a robot in the work cell. However economic feasibility requires to reduce the robotization costs. As it was pointed out earlier, the Factory-in-a-day project \cite{fiad} focuses on reducing the robotization cost by reducing the system integration cost and installation time. The key idea is develop generic and flexible robot solutions so that it can be quickly re-installed and configured to another temporary product line. To achieve this flexibility and maintain acceptable levels of productivity, in the Factory-in-a-day approach we propose to automate the easy 80\% of the tasks and leave the hard 20\% for human co-workers. 

Robot manipulators provide power, repeatability and extended work-space while the human operators provide flexibility and problem solving capacity. In addition, fenceless collaborative robots save space and installation cost. However, this approach requires a very high level of safety and agility; the robots should be aware of any obstacle, including dynamic obstacles such as its humans co-workers, and be able to move to avoid contact. Whereas current co-bots guarantee safe contacts, they degrade the performance of the work cell because of stopping the production. Collision avoidance using skin sensors data locally combined with point cloud data based replanning is the main idea of the framework. This framework being a breakthrough development of the Factory-in-a-day project allows robot arms to be aware of all the (dynamic) obstacles in their environment and respond reactively by moving around these obstacles while still continuing their work. In robot applications, path planning and motion control are usually formalized as separate problems though both of them fundamentally solves what a robot should do next. High dimensional configuration spaces, changing environment and uncertainties does not allow to plan real time motion ahead of time requiring a controller to execute the planned trajectory. The fundamental inability to unify both these problems has led to handle the planned trajectory amidst perturbations and unforeseen obstacles using various trajectory execution and deformation mechanisms. Designing an appropriate architecture to handle the information flow between the control and planning components is not so trivial. This makes dynamic collision avoidance a challenging and a completely unsolved problem in robotics. In our framework, we simplified the problem by combining both the individual advantages of a point cloud data based path planner and hierarchical task based reactive controller, depending on the status of the task. 

The chapter is structured as follows. In section \ref{sec:ca}, the developments in collision avoidance until now are summarized to illustrate the relevance of the proposed approach and to discuss the merits \& demerits of the proposed framework in the later part of the chapter. This is followed by the section \ref{sec:framework} presenting the framework, the individual components in detail and the technical connection between them in a manipulation scenario. The section \ref{sec:sot} presents the main contribution which is the reactive motion control part of the framework by modeling collision avoidance constraint as an inequality task fed to 'Stack of Tasks' controller to generate collision free motion behaviors. The section \ref{sec:sot} demonstrates collision avoidance using the proposed methodologies in both simulation and real robots. The section \ref{doa:conclusion} discussed the merits \& demerits of the proposed methodologies along with conclusive remarks.  