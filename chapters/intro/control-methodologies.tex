\section{Control Methodologies in Humanoid Robotics}
\label{sec:control_methods}
Control methodologies in robotics generally define a control law that allows feasible motion generation with respect to both the robot and environmental constraints to achieve one or multiple desired tasks. There are a variety of methods to generate motion depending upon the robot, the environment and complexity of the task. The control of humanoid robots is quite specific and challenging because of the kinematic redundancy and the dynamic complexity of the system. They have a non-trivial kinematic tree structure with an essential need to stabilize the position of its center of mass(CoM) with respect to the feet at every moment while executing other tasks. In simple words, a humanoid robot cannot walk or run without knowing how to balance itself while in motion. 

Another challenging aspect to be considered here is that the dimension of task space does not equal the dimension of the actuators. Typical tasks consist in controlling the position and orientation of an end-effector (i.e. 6 dimensions), while a humanoid robot has more than 20 degrees of freedom. An inherent task-joint space mapping exists in the form of a trained nervous system in human beings but the robots need a very good dynamic model and appropriate cost functions to make deterministic interactions with the world. Humanoid robots are under-actuated systems which means their pose cannot be controlled directly but by a consequence of commanding appropriate trajectories in joint configuration space. In the following we discuss the state of the art 
approaches used in generating motions.

\subsection{Motion Planning}
Motion planning in robotics refers to the process of searching discrete and feasible motion sequences to achieve a desired task usually satisfying safety constraints and optimal criteria relevant to the task. A motion planning algorithm can be used to plan motion for a variety of tasks ranging from simple arm manipulation for 6 degree of freedom (DoF) robot arms \cite{donald1987search,lozano1987simple} to advanced walking pattern generation for legged robots \cite{kajita2003biped,huang2001planning,harada2006analytical}. The basic idea is to produce continuous motion sequences that connect a start and a goal configuration of the robot while avoiding collision with the obstacles and satisfying certain criteria specific to the task. 
\paragraph{Work Space}
The \textit{Work Space}(\WS) is the set of all robot positions that define the working volume of the robot end effector \cite{cao2011accurate}. This idea can be extended to mobile robot navigation or humanoid locomotion though the definition restricts to end effectors of robot manipulators. The robot and the obstacles are defined in the (\WS) but the motion is always computed or represented in the configuration space, which usually has higher dimension.  

\paragraph{Configuration Space}


The \textit{Configuration Space}(\CS{}) is the set of all configurations a robot can possibly attain. The \CS{} for a robot with \textit{n} degrees of freedom, is a manifold \M{} of \textit{n} dimensions with all robot configurations $q\in$\M{}. The important aspect is that the planning problem in a \WS{} of \sethree{} is transformed to a planning a point motion in its corresponding \CS{}. \CSobst{} represents the robot configurations that are self-colliding or in collision with the environment while \CSfree{} represents its complement. These subspaces allow the planning algorithm to search for a feasible path connecting the start and the end configuration avoiding collisions. 


\paragraph{Algorithms}Over the last 30 years, there has been a lot of research done in motion planning due to the variety of applications . A planning  algorithm is said to be complete if it can find a plan for all the instances of a problem when at least one exists, or report a failure if none exists. Computational complexity  is used to assess the performance of complete planning algorithms. Incomplete planners are not reliable though they are quite effective sometimes in practice. The algorithms can be categorized into \textit{deterministic} and \textit{sampling} based algorithms.
\begin{itemize}
\item \textit{Deterministic Algorithms}: These planning algorithms rely on a deterministic function to compute the path, always resulting in the same motion plan given the same planning request at any instant. Geometric approaches such as visibility graphs, cellular decomposition, Voronoi diagrams and Canny's algorithm compute the shape of \CSfree{} and its connectivity using graphs or road maps \cite{toth2004handbook,canny1988complexity} whereas Grid based approaches represents the configuration space in grids. They use search algorithms(such as $A^*$) to find a path based on the discrete number of actions within \CSfree{} region. These algorithms in general are computationally expensive for high dimensional configuration spaces. Potential field based methods \cite{barraquand1992numerical,koren1991potential} relies on artificially creaing an attractive potential field for the goal and a repulsive field to plan the instantaneous path. The approach is very efficient but it suffers from local minima. Reward based algorithms search for a path that maximizes the cumulative reward constrained by positive reward for reaching a goal and negative reward for collision with obstacles. Markov Decision Processes framework is used in most of the reward-based algorithms generating optimal path \cite{spaan2004point}. 
\item \textit{Sampling based Algorithms}: Sampling based algorithms approximate the connectivity of the feasible configurations in \CSfree{} by random sampling collision free configurations from \CS{} \cite{hudson1997v,gottschalk1996obbtree,hsu1997path}. \textit{Rapidly-exploring Random Trees}(RRT) is a quite popular algorithm which uses Voronoi bias to explore the free configuration space and grows a tree by sampling randomly at every iteration to connect the start and the goal point in \CSfree{}. In a slightly modified version referred to as \textit{Bi-directional RRT}, trees are grown from both the start and goal simultaneously to accelerate the process of the search \cite{kuffner2000rrt}. \textit{Probabilistic Roadmaps}(PRM) randomly sample from the \CS{} and connections are made with the neighbors generated by \textit{k} nearest neighbors or using a local planner \cite{karaman2011sampling}. A road map is developed by adding more configurations and connections until it is dense enough to be used for a planning problem. A path is searched or queried on the generated roadmap using Dijkstra's shortest path algorithm. There are a variety of PRM extensions to achieve better performance \cite{geraerts2004comparative}. Variations such as PRM* and RRT* for generic use cases preserve the asymptotic optimality of the tree \cite{karaman2011sampling}. Kinodynamic RRT* has been proposed for systems with controllable linear dynamics \cite{webb2013kinodynamic} and RRTs have been extended to LQR-Trees in state space \cite{tedrake2010lqr} and LQR-RRT* for linearized systems \cite{perez2012lqr}. Another extension \textit{Transition based RRT} uses stochastic optimization for computing potential states \cite{jaillet2008transition}. Sampling based algorithms deal with high dimensional configuration spaces and are probabilistically complete which means the probability that they fail to provide a solution approaches zero when more time is spent though there is no guarantee if a solution really exists or not.
\end{itemize}


Trajectory optimization is used to reduce the path length while preserving its validity. There are a variety of methods to optimize the trajectory such as greedy optimization \cite{thrun2002probabilistic}  which connects the start and the nearest node configuration that avoids obstacles in an iterative sequence until it reaches the goal configuration. 



\paragraph{Planning in Humanoid Robots}


As discussed before, Humanoid robots are under-actuated and balancing is essential to perform any other meaningful tasks. The motion planning algorithms described previously search for a collision free path considering only the geometry but do not really consider the effects of joint configurations on the whole robot itself. RRT algorithms are able to generate collision-free trajectories but do not optimize their smoothness or the associated control effort. There is a certain amount of control necessary to deterministically choose the right trajectories for tasks specific to a robot and the environment. So in humanoid robots or any polyarticulated system, inverse kinematic/dynamic or optimal control methodology can be used to generate trajectories under constraints. For instance, in humanoid robots it could be necessary to satisfy balance constraints under multiple contact while walking up the steps. In \cite{zhang2014motion}, predefined motion primitives guide the planner to generate natural trajectories.

With regard to Humanoid walking, deterministic planning approaches use a foot transition model with dynamic considerations and the trajectory smoothness during posture transitions \cite{chestnutt2005footstep,ayaz2007human} though there is no guarentee of completeness. Instead probobalistically complete methodologies like RRT can be used to search in discrete footsteps space \cite{perrin2012fast,xia2009random}. Goal-directed approaches such as in \cite{hornung2013search} ensures the path within bounded limits set by the optimal control solutions. Environment features such as contact points can also be used to guide planning \cite{bouyarmane2012dynamics,Escande2013}. There are also some approaches that decompose the problem from higher dimensions into smaller dimensions and are successively solved \cite{zhang2009motion,yoshida2008planning}. Planning is also done on constraint sub-manifolds within \CS{}. Approaches such as in \cite{bretl2006motion,hauser2010multi}, planning is done on union of sub-manifolds defined by balance constraints and leg positions for robots on uneven terrains. In \cite{berenson2011task}, they use jacobian based methods 
to add end effector pose constraints in a \textit{Constrained Bidirectional} RRT planner. In \cite{dalibard2013dynamic}, RRT is adapted within a random diffusion framework to generate statically stable trajectories. In recent times, path planning is combined with optimal contol to generate motion in cluttered environment \cite{el2013optimal}.

\subsection{Kinematic Control}
\paragraph{Basics of Robot Kinematics}
Kinematics is the study of movement of kinematic chains considering the geometry while ignoring the dynamic properties such as mass, inertia of the system and forces or torques to generate motion. The relationship between position, velocity and acceleration of the links of the system with respect to its kinematic connectivity and dimensionality is studied to control the movement of the system in joint space. The joint space of a robot with \textit{n} DoF is an \textit{n} dimensional manifold $\mathcal{Q}$ with all possible joint values. But in robotics, the pose of certain points are important for control in the task space. These point are generalized to \textit{Operational Points} \cite{Khatib1987}. The variation in an operational point x can be represented by a twist $\xi$ comprising linear and angular velocities \cite{featherstone2008rigid,Murray1994}. The kinematic control problem can be defined in four different ways:
\begin{itemize}
    \item \textit{Forward Kinematics}: Given a joint configuration $q \in \mathcal{Q}$, it involves finding the pose of an operational point $x \in SE(3)$ described by $x = f(q)$ such that $f:\mathcal{Q} \rightarrow SE(3)$. Denavit-Hartenberg(DH) parameters \cite{hartenberg1955kinematic} introduced in \cite{craig2005introduction} is quite a popular method used to represent the forward kinematics for serial manipulators though they are not suitable for closed kinematic chains and robot calibration \cite{khalil2004modeling,everett1987kinematic}. Other approaches include the \textit{Khalil-Kleinfinger} notations for closed loop robots \cite{khalil2004modeling}, \textit{Hayati-Roberts} coordinates for avoiding singularities \cite{hayati1985improving, roberts1988new}, screw theory modeling \cite{tsai1999robot}, product of exponentials(PoE) \cite{park1994computational} and methods 
    specific to humanoid robots as in \cite{kajita2005humanoid}.

    \item \textit{Forward differential kinematics}: Given a joint configuration variation $\dot{q} \in T_q(\mathcal{Q})$, it involves finding the twist of an operational point $\xi \in se(3)$ described by  $\xi = J_o(q)\dot{q}$ such that  $J_o:T_q(\mathcal{Q}) \rightarrow se(3)$ where $J_o$ is the tangent or geometric jacobian \cite{spong2006robot,Khatib1987} and $T_q(\mathcal{Q})$ is tangential to $\mathcal{Q}$ at $q$.
    \item \textit{Inverse Kinematics}: Given a pose $x \in SE(3)$ for an operational point, it involves finding the joint configuration
    $ q \in \mathcal{Q}$ described by $q = f^{-1}(x)$ such that $f^{-1}:SE(3) \rightarrow \mathcal{Q} $. The \textit{Inverse Kinematics} problem can be solved analytically for certain kinematic structures either using algebraic approaches as in \cite{paul1robot} \cite{RaghvanBoth1993inverse} or geometric approaches as in \cite{paden1985kinematics, peiper1968kinematics}.

    \item \textit{Inverse differential kinematics}: Given a twist $\xi \in se(3)$ for an operational point, it involves finding the 
    joint velocities $\dot{q} \in T_q(\mathcal{Q})$ described by $ \dot{q} = J_o(q)^{+}\xi$ where $J_o^{+}$ is a pseudo inverse such that $J_o^{+}:se(3) \rightarrow T_q (\mathcal{Q})$. It is solved analytically in \cite{Chiaverini1994,Chiaverini1997,siciliano1999tricept}  

\end{itemize}

\paragraph{Kinematic Redundancy}
The kinematic control involves reaching a reference point or tracking a trajectory in $SE(3)$ for one or more operational points by searching for the appropriate instantaneous joint configurations  $q(t)$. Such goals are called tasks. When the dimensions of the robot $n$ exceeds the task DoF $n_t$, the robot is said to be kinematically redundant with $n − n_t$ being the degree of redundancy with respect to task \cite{nakamura1990advanced}. Though it provides flexibility in the joint space to manage the constraints effectively, it is quite complex to handle the inverse kinematics in a multiple task scenario \cite{siciliano1991general}. Closed form solutions are not possible in redundant robots and choosing a solution that solves both the main and the complementary tasks as much as possible is essential. For instance, it is expected from a robotic arm to manipulate an object while avoiding collisions with the environmental obstacles reactively. Numerical approaches are used to solve these multi-task control problems on redundant robots.

Numerical methods formulate \textit{Inverse Kinematics} as a constrained optimization problem either globally or locally. Global methods search for an optimal path for the entire trajectory which is computational complex \cite{baillieul1990resolution}. Local methods solve them differentially computing locally optimal $dq$ for a small change $dx$ to generate the joint space trajectory $q(t)$. \textit{Resolved Motion Rate Control} in \cite{Whitney1969} finds the $\dot{q}$ by solving the system: $\dot{x} = J(q) \dot{q}$. Damped pseudo inverses \cite{nakamura1986inverse} are used to avoid singularities and inversion issues in redundant robots when the Jacobian is not a full row or column rank matrix. 

 A more generic solution solves each task by projecting onto the nullspace of the Jacobian $J$ supplying certain degrees of freedom to the specific task \cite{Liegeois1977}. There are two ways to carry out the projection systematically.

 \begin{itemize}
     \item \textit{Task Space Augmentation} uses weighting to modulate the task space by constraining the joint space\cite{sciavicco1987solving}. Task conflicts are managed by assigning weights to each task making a compromise between the goals of the tasks.  Proper tuning specific to the scenario is essential for this methodology. \textit{Extended Jacobian Matrix} helps in handling the conflict between tasks by zeroing down the projection of the cost function gradient on the null space of the Jacobian \cite{baillieul1985kinematic}.

     \item \textit{Task Prioritization} strictly prioritizes the projection hierarchically by ensuring that the lower priority task do not affect the tracking error of the higher priority task \cite{nakamura1987task}. A systematic framework for a multi-task scenarios is proposed in \cite{siciliano1991general} which algorithmically generates $\dot{q}$ to minimize the task error in a prioritized way. A solution in \cite{Baerlocher1998} is used to speed up the computation of null space projector. 
     
 \end{itemize}

\paragraph{Redundancy Resolution in Humanoid Robots}
Since humanoids and redundant robots have a tree like structure, closed form solutions are generated by treating the complete robot as a set of many kinematic chains as in \cite{ali2010closed,nunez2012explicit}. These approaches are quite complex and instantaneous inverse kinematic solvers are usually preferred. Linearizing the problem on humanoid robot, produces infinite solutions providing the possibility to perform multiple tasks. \textit{Task Space Augmentation} methods as in \cite{tevatia2000inverse} and \cite{salini2009lqp} suffer from task conflicts resulting in unsuccessful task execution whereas \textit{Task Space Prioritization} has a strict priority on resolving multiple tasks leading to locally optimal results.
 The method is also computationally less expensive and are used in common. Several approaches have been developed for multiple equality constraints at the kinematic level \cite{yoshida2006task,mansard2007task,gienger2005task}. For inequality tasks, sequential quadratic programming is used to solve a cascade of tasks \cite{kanoun2009prioritizing}. A much more efficient implementation using complete orthogonal decomposition
is proposed by \cite{Escande2014} which the state of the art solver Stack of Tasks used to solve a hierarchical quadratic program. \cite{jarquin2013real} also proposes a solution for a smooth transition in control when the priorities are interchanged.  


\subsection{Dynamic Control}

\paragraph{Basics of Robotic Dynamics}
Dynamics is the study of the dynamic relationship between the motion of the kinematic chain and the generalized forces acting on the system to make movement. The generalized forces are the external forces applied to the system which include the joint torques for rotational joints, joint forces for prismatic joints, and also the force due to external contacts. This relationship allows to control the robot in a dynamic level enabling a better interaction with the environment. Dynamic parameters such as length, mass, inertia of the links and the forces or torques acting on the mechanical system are considered. In a robot dynamic model, the motion is defined by joint  acceleration $\ddot{q}$ and operational point acceleration $\ddot{x}$ in the task space. 

\begin{itemize}
    \item \textit{Forward Dynamics} determines the acceleration of the robot in the joint space when generalized forces are applied in the joint space
    \item \textit{Inverse Dynamics} determines the generalized forces necessary in joint space to achieve a required acceleration in the joint space.
\end{itemize}

The two main approaches to model the robot dynamics are:

\begin{itemize}
    \item \textit{Lagrange Method} is energy based with dynamic equations in closed form. \cite{uicker1969dynamic,kahn1969near,bejczy1974robot} are such approaches in the robotics domain. It has a clear separation of each component but it is very expensive  for implementing control schemes. \cite{hollerbach1980recursive} presents an efficient formulation but still \textit{Newton-Euler Method} are preferred for faster computation.
    \item \textit{Newton-Euler Method} is a generalized force based and recursive in nature. The first of its kind is presented in \cite{orin1979kinematic}. It does not clearly separate components but it is computationally cheaper. \cite{Featherstone2009} explains the most common algorithms such as the composite-rigid-body algorithm (CRBA), the articulated-body algorithm (ABA), the recursive Newton-Euler Algorithm (RNEA). 
\end{itemize}
\cite{spong1992remarks} uses an hamiltonian approach for the analysis of the robot dynamics and there exist certain numerical methods to integrate hamiltonian equations efficiently. Alternatively, \textit{Centroidal dynamics} \cite{orin2013centroidal,orin2008centroidal} models the dynamics of the CoM of the robot capturing the constraints imposed by contact forces on the CoM making the dynamic model very simple. But it does not include joint position and torque limits which makes the model approximate failing to describe the angular momentum of the robot. Humanoids walk with very little angular momentum and is set to zero applying when \textit{Centroidal dynamics} .
In contrast to the classic joint space formulation, the operational space formulation \cite{Khatib1987} defines the motion using the task space acceleration, which requires the forces to be formulated as generalized forces in the task space.

\paragraph{Dynamic Control in Humanoid Robotics}
As discussed already, classical techniques are not appropriate for humanoid robots as the control scheme has to execute coordinated motion keeping into account the interaction forces exchanged with the environment along with balance all the time. Stability criteria that constrain the CoM or the zero moment point (ZMP) within the support polygon are often enforced in the controller \cite{wieber2002stability}. A constraint on the ZMP is equivalent to a constraint on the tangential moment at the contact surfaces. The interaction of legs on the ground along with other contacts of the robot body with the environment is also necessary. The contact forces are transformed to generalized forces on the joints which can be controlled only using joint torque control. Dynamic control is quite essential in ensuring motion feasibility, agility and balance in 
humanoid robots.

The operational-space inverse-dynamics (OSID), a generic framework  establishes the whole-body control considering balance, contacts and other constraints. \cite{Khatib2004} proposes a multi-task formulation with sequential projection on  the nullspaces of the tasks at each level. Strict hierarchy allows lower priority tasks not to affect the higher priority tasks. An equivalent approach \cite{mistry2010inverse} uses orthogonal decomposition and kinematic projections to simply control when switching contact constraints and avoids the inversion of mass matrix. \cite{Righetti2011a,Righetti2011} proposes an improved version constraining the ground reaction forces with the friction boundaries. Inequality constraints can not be added in this framework which is quite necessary to implement collision avoidance, joint limits and visual servoing task. OSID's are used within optimization based methodologies to find optimal solutions. Quadratic Programming (QP) based approaches are more popular for redundant systems which allows both equality and inequality tasks. Weighting schemes are used in a QP based approach which provides robust balance \cite{collette2008robust}. \cite{Salini2011} implements a weighting focused on the task prioritization. Decoupling of the dynamics in holonomic and non-holonomic part is used in a QP to generate feasible whole body trajectories \cite{bouyarmane2012dynamics,wieber2006holonomy}. \cite{herzog2013experiments} uses active set method to implement a torque control of the lower part of the robot. OSID in a strictly prioritized QP framework is implemented in \cite{Saab2013}. 


The previous approaches focused on the robot dynamics but the angular momentum was not controlled specifically which is an important part of motion in human beings doing agile and complex motion \cite{popovic2004angular}. \cite{kajita2003resolved} proposed to control the angular momentum of the robot using the inverse of the inertia matrix. \textit{Centroidal dynamics} controls the angular momentum as in \cite{hofmann2009exploiting} focusing on gait movements and \cite{moro2013attractor} where an attractor is used to control. Also \cite{wensing2013generation} uses conic optimization techniques to control the angular momentum generating complex motions.

\subsection{Optimal Control}
Optimal Control involves finding control laws for a given system such that certain optimality criteria are minimized. It is a problem of searching trajectories that satisfy the system dynamics and minimizing a cost function. In Robotics, it is also called as \textit{Trajectory Optimization} or \textit{Trajectory Filtering}.

\paragraph{Background}
Optimal Control is actually an extension of a theory of \textit{Calculus of Variations} that uses optimization methodologies to find control policies. The first ever optimal control solution was proposed for the brachystochrone problem in Bernouli's work. Though there were early contributions to the theory of optimal control by Newton, Euler, Leibniz, Jacobi, Hamilton, Bolza and many others, the formalization began to take shape with the introduction of Linear Quadratic Control(LQC) problem minimizing a squared error of an expected value of a random variable x \cite{wiener1949extrapolation}. The next milestone was the birth of Dynamic Programming (DP) to solve discrete time optimal control problems which reduces the search to Hamilti-Jacobian equation. The formulation of the Pontryagin maximum principle \cite{pontryagin1987mathematical} completely formalize the problem based on the calculus of variations considering  pathwise constraints on control values of the functions. \textit{Linear Quadratic Regulator (LQR)} and the \textit{Linear Quadratic Gaussian (LQG)} are formulated to design optimal control policies in \cite{Kalman1960} which marks another important formulation in optimal control.

\textit{Model Predictive Control (MPC)} also called as Receding Horizon Control is a popular automatic control methodlogy in industries which uses the optimal control solution at each instant over a time horizon called the prediction horizon \cite{richalet1978model}. Task specific goals are specified by simple penalty functions in the optimal control which synthesizes the motion behaviour and control laws. The controller predicts the futures states under certain criteria which lets them find the control values to achieve the appropriate current state avoiding extensive exporation. \textit{Differential Dynamic Programming(DDP)} is a numerical scheme quite efficient for direct implicit optimal control generating local trajectories \cite{jacobson1968new}. A hybrid method uses constant local controllers \cite{atkeson1994using} which is improvized in \cite{atkeson2003nonparametric} using second order local models to make linear controllers.


\paragraph{Optimal Control in Humanoid Robotics}
For humanoid robot walking, the \textit{Operational Space Inverse Dynamics} or \textit{Inverse Kinematics} cannot really handle the accelerations of the CoM properly thus generating very conservative or over restricted motion. Optimal control can find 
trajectories from initial to final posture specified as one complete objective or many objectives with certain constraints. Optimal control can generate proper and powerful trajectories for CoM unlike IK or OSID. It is quite equivalent to a classical walking pattern generator in addition to the ability to incorporate whole body motion. MPC's are a quite relevant control scheme in humanoid robotics and control solutions as in \cite{kajita2003biped,herdt2010online} uses a dedicated Linearize Inverted Pendulum model to predict the future state of the system for controlling dynamic stability. Though it is quite complex and the models have to be created for every new case, optimal control is very appropriate and takes care of all the constraints at the same horizon. One problem with using optimal control in humanoid robotics is the computational time because of more DoFs and it requires a faster response when controlled in dynamic level.

In multiple task scenarios, weighted average of tasks can be used to influence the priority by choosing right weights for each task in the set \cite{dimitrov2011sparse}. It obviously introduces task conflicts if the relative difference of the weights are not significantly high. Large weights can also produce numerical errors resulting in pracitcal complexity to implement such control schemes. In \cite{del2014prioritized}, the optimal control is strictly prioritized avoding such issues. MuJoCo is a state of the art simulator that uses MPC to generate humanoid trajectories with high computational speed with respect to dynamic derivatives\cite{todorov2012mujoco}. Behaviours like getting up from the ground and avoiding disturbances were generated in this simulator \cite{tassa2012synthesis}.  The use of environment increases the controllable space to successively acheive the goals as in \cite{lengagne2013generation} generating non-coplanar contact motions. Parkour motion\cite{dellin2012framework}, kicking a ball \cite{miossec2006development} and  lifting weights \cite{arisumi2008dynamic} shows some application of optimal control in generating complex behaviours.

Variants of DDP exist with respect to optimal control in humanoid robots. Control-limited DDP allows adding inequality constraints as proposed in \cite{tassa2014control}  has been applied on a humanoid robot in simulation. Square-root DDP proposed in \cite{geoffroy2014inverse} shows that IK and OSID are special cases of optimal control without preview horizon. DDP is also used in generating sample trajectories to learn a neural network based guided policy to avoid local minima in \cite{Levine2012}. Behaviors such as walking, running, hopping and swimming are generated using this approach. Robust walking behaviours were generated using DP in \cite{whitman2013coordination} relying on multi model and learning based DP variants. Steep climbing motion is generated in \cite{Noda2014}
which uses Body Retention Load Vector to model the physical constraints. Optimal control has also been treated as an offline control problem in \cite{schultz2010modeling} optimizing energy based criteria using multiple shooting approach to generate running behaviors. Walking motions are generated without using a pattern generator but just by optimizing joint velocities, torques or ZMP constraints \cite{el2013optimal,koch2012studying}. Inverse optimal control involves learning the criteria to generate observed motions in a dynamic system. Human locomotion is studied using inverse optimal control in one such work \cite{mombaur2010human}. Also human running is studied in different terrains and distirubances in \cite{park2013inverse}


Motion planning is also combined with optimization for collision free navigation in a cluttered environment. Motion planners for high DoF systems generate trajectory in two stages: planning and optimization. There are three well known and similar techniques in this domain. CHOMP (Covariant Hamiltonian Optimization for Motion Planning) uses covariant gradient descent technique in the solver resulting in a planning algorithm completely relying on trajectory optimization \cite{zucker2013chomp}. Starting with a naive trajectory, CHOMP optimizes the dynamics of the trajectory while reacting to the obstacles in the environment. STOMP (Stochastic Trajectory Optimization for Motion Planning) is very similar except for the use of trajectory stochastic perturbations to generate trajectories without computing the jacobian \cite{kalakrishnan2011stomp}. ITOMP (Incremental Trajectory Optimization for Real-Time Replanning in Dynamic Environments) can produce suboptimal solutions because of the time constraints on the solver and the trajectory is incrementally updated online \cite{park2014high}.

High dimensionality is challenging in humanoid robotics to get optimal control working in real time. The state space gets large due to high number of DoFs and it is not easy to explore all of it apriori and generate appropriate trajectories for every cases. Current optimal control solutions are time consuming and encounters numerical problems which is sill an open issue in robotics.