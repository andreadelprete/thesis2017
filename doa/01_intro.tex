
\section{Introduction}
A desire for robotic solutions, particularly in the Small and Medium scale Enterprises (SMEs) is becoming increasingly prominent. Automation and robotics promise to deliver reduction on production costs and increase in productivity. However, traditional automation implies an investment prohibitive for SMEs, whose activities mainly involve small batches of production and high variety of products, for example, due to a seasonal nature of their operations. Concretely, tasks such as assembly, machine filling or packaging, can be automated with a robot in the workcell. However economic feasibility requires to reduce the robotization costs. The Factory-in-a-day project \cite{fiad} tries to reduce the robotization cost by reducing the system integration cost and installation time. The key idea is that the robot solution is flexible so that it can be quickly re-installed and configured to another temporary product line. 

To achieve this flexibility and maintain acceptable levels of productivity, in the Factory-in-a-day approach we propose to automate the easy 80\% of the tasks and leave the hard 20\% for human co-workers. Robot manipulators provide power, repeatability and extended work-space while the human operators provide flexibility and problem solving capacity. In addition, fenceless collaborative robots save space and installation cost. However, this approach requires a very high level of safety and agility; the robots should be aware of any obstacle, including dynamic obstacles such as its humans co-workers, and be able to move to avoid contact. Whereas current co-bots guarantee safe contacts, they degrade the performance of the work cell because of stopping the production. This is one of the breakthrough innovations of the Factory-in-a-day project, robot arms that are aware of all (dynamic) obstacles in their environment, and that respond by moving around these obstacles while still continuing their work.

In this chapter,  Section \ref{sec:framework} presents the framework, its individual components and the connection between them.  Section 2 presents the SOA of collision avoidance.




