\section{Conclusion}

This chapter presented the technologies that have been developed in the FiaD
project to augment collaborative robot manipulators with dynamic obstacle
avoidance. All these technologies: a proximity-sensing robot skin, a reactive
path planning solution and a robot motion control strategy, have been validated
in laboratory prototypes. Also, a preliminary prototype of an integrated
solution based on these technologies has been tested in simulation. With the current promising results, we are currently working on a robotic system prototype (based on the setup in Fig. \ref{fig:TUDSetup})

% work in progress%=CITE SUPPORTING EVIDENCE THAT IT WORKS=.
The integration and installation of advanced functionalities such as the dynamic
obstacle avoidance solution presented poses three main challenges from the
software point of view. The first is the integration of different components such as the skin driver, path planner and robot motion control. We address this challenge by adhering to the software development paradigm of the ROS-Industrial initiative. All the components discussed in this paper have been successfully integrated with ROS.

A second challenge is the quality assurance and robustness of the integrated robot software. This is crucial in production environments, and is specially important in collaborative applications, where safety needs to be guaranteed. For this purpose an Automated testing Framework (ATF) has been developed \cite{Weisshardt-2016} as a part of the FiaD prohect, which
allows for the systematic testing of robot software components, which includes unit
 testing, simulation-in-the-loop testing and eventually hardware-in-the-loop testing.
The tests can be automated and integrated in a centralized continuous
integration system. Preliminary test have already been conducted with the
components of the robot software system of this work, and the integrated
prototype applications will be tested with ATF.

Finally, the third challenge is the deployment of the software. One of the main barriers to transfer solutions based on robot frameworks such as ROS to industry, and specially SMEs, is how cumbersome it is
to deploy. As a part of the FiaD project, a Robot deployment toolbox has been developed \cite{Ludtke-2017}, based on ROS, which can also be integrated with ATF.
The deployment tools will also be evaluated on the RBE17 prototype.


% The paper proposed a creative method to execute a trajectory robust to run-time inequality constraints without compromising on the final goal of a scenario. The method uses 'Stack of Tasks', a jacobian control framework employing the state of the art HQP solver for both equality and inequality constraints. The creativity lies on the way the hierarchical nature of the solver and flexible task definition of the framework is exploited to combine an intuitive main goal definition and reactive trajectory execution to realize a scenario successfully robustly. Experiments were done to verify the validity of the method on a PR2 robot with a skin sensor(in simulation) mounted on its forearm. Future work will focus on generalizing the method for various kind of tasks and extending the framework to systematically design tasks to be added in the controller stack specific to possible desired scenarios. The methodology has a significant potential to be used for a robot setup with full body skin sensors.