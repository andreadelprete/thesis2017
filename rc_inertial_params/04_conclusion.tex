
%!TEX root =  ../root.tex
\section{Conclusions}
\label{sec:conclusions}
This chapter presented a novel optimization-based inverse-dynamics controller that can balance a legged robot despite bounded uncertainties in its inertial parameters. The controller is based on the state-or-the-art control framework Task-Space Inverse Dynamics. In particular, this work is based on the capture-point inequalities~\cite{Ramos2014a}, which can be included in the controller formulation to ensure the balance of the robot on a level ground. We extended these capture-point inequalities to be robust to bounded uncertainties in the inertial parameters of the robot. The resulting optimization problem is still a Quadratic Program with the same number of variables and inequalities. Moreover, the time required for the additional computation of the robust controller is negligible in this context (i.e. a few microseconds).

We tested the robust controller in simulations with the HRP-2 robot, trying to reach a target position with its right end-effector while balancing. We performed several batches of 100 simulations each, introducing different errors in the inertial parameters and varying the position of the target position and the required speed of motion. Comparisons against a classic TSID controller have shown impressive improvements in terms of fall prevention.

\subsection{Future Work}
In the derivation of the robust controller we saw that the inertial parameters appear in different terms of the optimization problem.
In this preliminary work we focused only on how the uncertainties affect the CoM position.
We believe it should be possible to extend this analysis to the other terms in the capture-point inequalities: CoM velocity, CoM altitude, CoM Jacobian and its time derivative. 
Extending it also to the mass matrix and the bias forces is an interesting future direction, but it seems more challenging because of nonlinearities.

Another issue of the presented approach is that it is rather conservative. As we saw in Test 1, this can lead to poor performance, which can be unacceptable on a real system. Modeling uncertainties with probability distributions (rather than with polytopes) may lead to a less conservative behavior of the system, and it is thus an interesting future direction. In our previous work~\cite{DelPrete2015b} we presented another robust controller, which was robust to joint-torque tracking errors.Integrating the two controllers together seems to be feasible and it would provide robustness to both kinds of uncertainties. In this preliminary work we focused on simulations to validate the controller formulation and to test it with different parameter errors. Of course, we plan also to test the generated movements on the real HRP-2 robot, to quantify how much it can benefit from this robustness.
